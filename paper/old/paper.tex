\documentclass{article}
\usepackage{a4}
\usepackage{rotating}
\usepackage{epsfig}

\newcommand{\degrees}{\mbox{${}^{\circ}$}}

\title{The ups and downs of protein topology; rapid comparison of protein
structure} 
\author{Andrew C.R.\ Martin}

\begin{document}
\maketitle

\begin{abstract}
Protein topology can be described at different levels. At the most
fundamental level, it is a sequence of secondary structure
elements (``primary topology string''). Searching predicted
primary topology strings against a library of strings from known
protein structures is the basis of some protein fold recognition methods.
Here a method known as TOPSCAN is presented for rapid comparison of
protein structures. Rather than a simple 2-letter alphabet (encoding
strand and helix), more complex alphabets are used encoding direction,
proximity and accessibility in addition to secondary structure.

The algorithm is extremely fast, with a scan of a large TIM-barrel
domain against a library of more than 2000 secondary structure strings
completing in around 30 seconds on a 200MHz Pentium class processor.

It has been used together with the SSAP structure-comparison algorithm
as part of a system for classifying a novel protein structure within
the CATH classification of protein domain folds.

\end{abstract}

\section{Introduction}
As new structure data becomes available, the classification of protein
folds is becoming more and more important. When a new structure is
solved, one wishes to ask the question of whether this fold has been
seen before. If the fold is one of the commonly occuring superfolds
(e.g.\ immunoglobulin-fold, TIM barrel, $\alpha\beta$-plait, Rossman
fold), then it can generally be recognised by eye, but the less common
folds are more difficult to recognise. Automated servers to answer
this question have recently been developed and rely on structure
comparison programs.

Unfortunately, detailed automatic structure comparison is a
time-consuming process. The double-dynamic programming algorithm used
in SSAP at the atomic coordinate level is particularly computer
intensive making it impractical to run a full scan of a protein of any
size.  An experimental ``CATH-server'' at UCL has made use of sequence
screening and a hierarchical expansion scheme through the
representative levels in the CATH domain classification database to
reduce the search, but in the worst case scenario, it could still be
necessary to scan against more than 2000 near-identical sequence
representatives. We estimate approximately 2 weeks of computer time
would be required to scan a large protein domain such as a TIM barrel
in this way and this is clearly impractical for use as a server.

When a sequence for a protein of unknown structure does not show
obvious sequence homology with a protein of known structure, fold
recognition is commonly used to give clues as to the three-dimensional
structure. There are two common approaches to this problem: one is
``threading'' which assesses how well a sequence is accomodated within
a three-dimensional structure; the other is alignment of a predicted
sequence of secondary structure elements against a fold library
encoded in this form.

This description of a protein structure as a sequence of secondary
structures is the true topology of the protein. The dictionary
definition of topology is the factors which remain unchanged as an
object undergoes a continuous deformation. i.e.\ if one imagines a
protein as a piece of string with beads representing the secondary
structures along the string, straightening the string or folding it up
will not change the topology. Here, we describe this as the ``primary
topology''. A primary topology string is a sequence of E and H
characters (representing $\beta$-strand and $\alpha$-helix in Kabsch
and Sander, DSSP, notation).

However, when we talk about protein topology, this is generally not
what we mean. Rather, using the CATH hierachical classification of
protein structure, the ``topology'' level is actually describing the
topology of a given three-dimensional architecture. i.e.\ given a
particular spatial arrangement of secondary structure elements, the
topology describes how these elements are connected. What is commonly
referred to as the topology of a protein is, in fact, its
three-dimensional fold. Here, we describe this as the ``tertiary
topology''. 

Fold recognition methods which use primary topology strings are
therefore making the assumption that the tertiary topology can be
predicted from the primary topology. We present an analysis of the
occurrence of given primary topologies in different tertiary
topologies. 

Here the use of topology strings is extended to the comparison
of three-dimensional structures. Given the extra information available
within a three-dimensional structure, we are able to introduce an
intermediate level of topology ``secondary topology'' which contains
additional information (secondary structure element direction,
proximity and accessibility) to improve the mapping between these
lower levels of topological description and ``tertiary topology''
(i.e.\ protein fold).

TOPSCAN was initially developed as a simple method to reduce the
search space for the CATH-server. In that inplementation, the seondary
topology strings contain only the directional information.  It is used
to sort domains and SSAP is then used to make the final selection
working down the TOPSCAN-sorted list. In this way, we can be sure that
we will not miss any hits which might otherwise have been found using
a full search with SSAP alone.


%%%%%%%%%%%%%%%%%%%%%%%%%%%%%%%%%%%%%%%%%%%%%%%%%%%%%%%%%%%%%%%%%%%%%%%%%
\section{Methods}
TOPSCAN reduces a protein to a topology string which can represent the
structure as a string of letters encoding simply the primary topology
(a 2-letter alphabet) or the seondary topology using various
additional data from the 3D structure. For example, by encoding
direction information with the secondary structure, a 12-letter
alphabet is used.

A simple Needleman and Wunsch dynamic programming algorithm is then
applied to compare two such strings and a score for the similarity is
calculated. Alternatively, a library of these topology strings may be
pre-built from the Protein Databank (or a representative subset) and a
structure may then be scanned against this library.

%11111111111111111111111111111111111111111111111111111111111111111111111
\subsection{Creating primary topology strings}
TOPSCAN enables secondary structure to be calculated from a
three-dimensional structure using either DSSP or STRIDE (for the
analyses presented in this paper, STRIDE was used).  From these
assignments, regions of $\beta$-sheet (Kabsch and Sander assignment,
E) and of $\alpha$-helix (Kabsch and Sander assignment, H) are
extracted. Only continuous regions of at least a specified number of
residues with the same assignment are selected.  This produces the
primary topology of the protein as a string of E and H characters.

%11111111111111111111111111111111111111111111111111111111111111111111111
\subsection{Creating secondary topology strings}
To increase the information content of the topology string, various
information from the three-dimensional structure can be incorporated
into a secondary topology string. We examime the application of three
of these.

%22222222222222222222222222222222222222222222222222222222222222222222222
\subsubsection{Secondary structure element direction}
The end-points of each secondary structure element in the primary
topology are found and the vector between them is calculated. The
direction of the vector is grouped into one of 6 classes depending on
the largest component of the vector (i.e.\ positive or negative $x$,
$y$, or $z$). This is equivalent to saying the element points up,
down, left, right, forward, or back. The encoding is summarised in
Table~\ref{tab:encoding}.

\begin{table}
\begin{center}
\begin{tabular}{llll} \hline
\multicolumn{2}{c}{Direction} & \multicolumn{2}{c}{Secondary Structure}\\ \cline{3-4}
          &         & strand & helix  \\ \hline
$+y$      & Up      & A      & G      \\
$+x$      & Right   & B      & H      \\
$-y$      & Down    & C      & I      \\
$-x$      & Left    & D      & J      \\
$+z$      & Back    & E      & K      \\
$-z$      & Forward & F      & L      \\ \hline
\end{tabular}
\end{center}
\caption{\label{tab:encoding}Encoding scheme used to represent
          secondary structure and direction information}
\end{table}


Two topology strings are compared using a simple Needleman and Wunsch
dynamic programming algorithm. A scoring matrix
(Table~\ref{tab:fullmatrix}) is used in the comparison based on the
scoring scheme shown in Table~\ref{tab:matrix}. The scoring scheme is
somewhat arbitrary, but appears to work well.  In essence, the same
secondary structure in the same orientation scores highest. Different
orientations score worse and different secondary structure types
achieve much lower scores.

The starting premise was that for the same type of secondary
structure, one wishes to assign 3 scores: 
\begin{center}
\begin{tabular}{rlll}
\multicolumn{2}{l}{Angle between vectors}&\hspace{1cm}& Direction  \\
            & $\Delta\le 45\degrees$     && Approximately the same \\
45\degrees  & $\Delta\le 135\degrees$    && Approximately right angles \\
135\degrees & $\Delta\le 180\degrees$    && Approximately opposite \\
\end{tabular}
\end{center}
\noindent Arbitrarily, these were assigned as scores of 10, 5 and 2
respectively. However, because of the boolean definition of whether a
vector is in a given quadrant, it is possible that vectors actually
point in very similar directions although they are in different
quadrants (for example one points at $+$89\degrees\ while another
points at $+$91\degrees). Picking pairs of random numbers between
0--90 and 90--180 and plotting the distribution of the differences
shows an upside down V shaped curve centred around 90\degrees
(Figure~\ref{fig:dist}). Two vectors in adjacent quadrants will
actually be within 90\degrees\ of one another 50\% of the time. The
actual score for adjacent quadrants is therefore the median of 10 and
5 (rounded up to 8).

\begin{figure}
\centerline{\epsfig{file=dist90.ps}}
\caption{\label{fig:dist}Distribution obtained by picking 100000 pairs
of numbers between 0--90 and 90--180 and plotting the differences.}
\end{figure}

When two vectors are assigned to opposite quadrants, they can never be
closer than 90\degrees. Picking pairs of random numbers between 0--90
and 180--270 gives an identical upside down V shaped distribution
centred around 180\degrees. The angle between two vectors can never be
greater than 180\degrees, so every angle $<$180\degrees\ observed
between vectors in opposite quadrants can also be seen in vectors
assigned to adjacent quadrants.  In opposite quadrants, the angle
between two vectors is $<$135\degrees\ (our cutoff for saying two
vectors are approximately at right angles) $12.5$\% of the time. We
therefore assign the score as $2+((5-2)\times 12.5/100) =
2\frac{3}{8}$ which we round back down to 2.

For different secondary structure types, we make the simple arbitrary
assignments of 3, 1 and 0.

%% %% %% %% %% %% %% %% %% %% %% %% %% %% %% %% %% %% %% %% %% %% %% %% 
%% This really should be 3,2,0 - simply by applying the rule to the  %%
%% previous scores of ((3/8) * (x-2)) : such that we scale them to   %%
%% have 3 as a maximum score and 0 as a minimum.                     %%
%% %% %% %% %% %% %% %% %% %% %% %% %% %% %% %% %% %% %% %% %% %% %% %% 

\begin{table}
\begin{center}
\begin{tabular}{rrrrrrrrrrrrr}
  & A & B & C & D & E & F & G & H & I & J & K & L \\
A &10 & 8 & 2 & 8 & 8 & 8 & 3 & 1 & 0 & 1 & 1 & 1 \\
B & 8 &10 & 8 & 2 & 8 & 8 & 1 & 3 & 1 & 0 & 1 & 1 \\
C & 2 & 8 &10 & 8 & 8 & 8 & 0 & 1 & 3 & 1 & 1 & 1 \\
D & 8 & 2 & 8 &10 & 8 & 8 & 1 & 0 & 1 & 3 & 1 & 1 \\
E & 8 & 8 & 8 & 8 &10 & 2 & 1 & 1 & 1 & 1 & 3 & 0 \\
F & 8 & 8 & 8 & 8 & 2 &10 & 1 & 1 & 1 & 1 & 0 & 3 \\
G & 3 & 1 & 0 & 1 & 1 & 1 &10 & 8 & 2 & 8 & 8 & 8 \\
H & 1 & 3 & 1 & 0 & 1 & 1 & 8 &10 & 8 & 2 & 8 & 8 \\
I & 0 & 1 & 3 & 1 & 1 & 1 & 2 & 8 &10 & 8 & 8 & 8 \\
J & 1 & 0 & 1 & 3 & 1 & 1 & 8 & 2 & 8 &10 & 8 & 8 \\
K & 1 & 1 & 1 & 1 & 3 & 0 & 8 & 8 & 8 & 8 &10 & 2 \\
L & 1 & 1 & 1 & 1 & 0 & 3 & 8 & 8 & 8 & 8 & 2 &10 \\
\end{tabular}
\end{center}
\caption{\label{tab:fullmatrix} Full scoring matrix for the 12-letter
alphabet encoding protein topology.}
\end{table}

\begin{table}
\begin{center}
\begin{tabular}{lll}\hline
                        & \multicolumn{2}{c}{Secondary structure} \\ \cline{2-3}
Orientation             & Same  & Different     \\ \hline
Same                    & 10    & 3             \\
Off by 1 quadrant       & 8     & 1             \\
Off by 2 quadrants      & 2     & 0             \\ \hline
\end{tabular}
\end{center}
\caption{\label{tab:matrix} Scoring scheme employed in the matrix for
the dynamic programming comparison of two topology strings.}
\end{table}



Because any pair of proteins is in an arbitrary relative orientation,
the definition of ``up'' in one protein may not correspond to ``up''
in the second. Therefore, one of the strings is permuted 23 times,
such that the dynamic programming algorithm comparison is performed a
total of 24 times (equivalent to the 6 sides of a cube, each of which
may be 4 ways up). Table~\ref{tab:permute} shows the modifications
made to the encoding to achieve rotations about $x$, $y$ and $z$ axes.

\begin{table}
\begin{center}
\begin{tabular}{llll}\hline
        & \multicolumn{3}{c} {Rotation axis} \\ \cline{2-4}
        & $x$           & $y$           & $z$           \\ \hline
Old     & ABCDEFGHIJKL  & ABCDEFGHIJKL  & ABCDEFGHIJKL  \\
New     & EBFDCAKHLJIG  & AECFDBGKILJH  & BCDAEFHIJGKL  \\ \hline
\end{tabular}
\end{center}
\caption{\label{tab:permute} Modifications made to topology strings to
        achieve rotations about the $x$, $y$ and $z$ axes.}
\end{table}




%22222222222222222222222222222222222222222222222222222222222222222222222
\subsubsection{Secondary Structure Element Proximity}
To add information about the packing of secondary structure elements,
the proximity of a secondary structure element to the element
precceding it is encoded. This is performed as follows (see
Figure~\ref{fig:proximal}).

\begin{figure}
\centerline{\psfig{file=proximal.eps,width=\linewidth}}
\caption{\label{fix:proximal}Calculation of proximity of a secondary
structure element to the preceeding element.}
\end{figure}

Given an element with endpoints C,D and a preceeding element with
endpoints A,B, the minimum distances between points A and B and the
line described by CD is calculated.  This is repeated with points C
and D to line AB. As each minimum point-to-line distance is calculated
the position on the line closest to the point is calculated. If this
point is between the two endpoints of the line and the distance is
$<XX$\AA, then we say that the element C,D is proximal to the
preceeding element and modify the encoding as shown in
Table~\ref{tab:encoding2}. 

\begin{table}
\begin{center}
\begin{tabular}{llll} \hline
\multicolumn{2}{c}{Direction} & \multicolumn{2}{c}{Secondary Structure}\\ \cline{3-4}
          &         & strand & helix  \\ \hline
$+y$      & Up      & M      & S      \\
$+x$      & Right   & N      & T      \\
$-y$      & Down    & O      & U      \\
$-x$      & Left    & P      & V      \\
$+z$      & Back    & Q      & W      \\
$-z$      & Forward & R      & C      \\ \hline
\end{tabular}
\end{center}
\caption{\label{tab:encoding2}Encoding scheme used to represent
          secondary structure and direction information where the
          element is proximal to the preceeding element.}
\end{table}




%22222222222222222222222222222222222222222222222222222222222222222222222





%11111111111111111111111111111111111111111111111111111111111111111111111
\subsection{Application within a classification protocol}
A CATHServer is being developed which allows the automatic
classification of a novel protein domain in the CATH hierachy.
TOPSCAN is employed within this protocol where no sequence hit is
found or where a sequence hit has been found but the SSAP score is
poor. TOPSCAN is then used to rank all the near-identical sequence
representatives (NREPs) for similarity to the probe domain. FAST-SSAP
(which works by comparing secondary structure vectors using
double-dynamic programming rather than by considering residue-level
environment vectors as is done with the full version of SSAP) is then
used to work down this ranked list. If a score better than 60 is
obtained with FAST-SSAP, the comparison is performed again using the
full version of SSAP. As soon as a full SSAP comparison achieves a
SSAP score of at least 70 with at least 60\% overlap of the
structures, the search is stopped and this structure is reported as
being the best hit. It is, of course, possible that there are better
hits farther down the TOPSCAN-ranked list, but this is a necessary
trade-off of computer time for absolute accuracy of results.





%%%%%%%%%%%%%%%%%%%%%%%%%%%%%%%%%%%%%%%%%%%%%%%%%%%%%%%%%%%%%%%%%%%%%%%%%
\section{Results and Discussion}
\input results.tex
\input best.tex
\input worst.tex
%\input bars.tex

\subsection{Asssignment of Secondary Structure}
The assignment of secondary structure is the critical first stage. We
have found that the DSSP algorithm can be over-sensitive to errors in
the structure. For example, NMR structures often show little secondary
structure in DSSP assignments whereas an intuitive visual inspection
shows substantial secondary structure. 

The STRIDE software from Argos and co-workers was designed to reduce
this sensitivity. However, using STRIDE in place of DSSP makes little
difference to the overall results (results not shown). The TOPSCAN
software allows either DSSP or STRIDE secondary structure assignments
to be used; for self-consistency, DSSP is used in the CATHServer
since this is used by the fast version of SSAP.

The simplified secondary structure assignment scheme used within
Rasmol actually appears to perform better in assigning secondary
structures in an intuitive manner and we are investigating the
possibility of extracting this code from Rasmol for stand-alone
secondary structure assignment.

\subsection{Effect of Secondary Structure Length}
We explored the effects of setting the minimum required number of
consecutive residues assigned to a given secondary structure to 3 and
4 residues (other values were tried, as were differing values for
helix and strand --- results not shown). Results for the best and
worst ranked true matches are shown in Tables~\ref{tab:best}
and~\ref{tab:worst} respectively. The observed differences are small,
but a setting of 3 for both helix and strand gives marginally improved
results. 

\subsection{Problems}
The algorithm has problems in discriminating, for example, a TIM
barrel from a Rossmann fold. These are topologically somewhat similar
--- in both cases, they consist of a core of $\beta$-sheet with
helices on the outside. In the case of the TIM barrel the
$\beta$-sheet is curved into a barrel where as the Rossmann fold has a
relatively flat sheet. Because the differences between the two occur
in a plane perpendicular to the direction of the secondary structure
elements, the current method does not distinguish reliably between these
architectures. 

%% %% %% %% %% %% %% %% %% %% %% %% %% %% %% %% %% %% %% %% %% %% %% %% 
%% Give examples of topology strings for TIM and Rossmann            %%
%% %% %% %% %% %% %% %% %% %% %% %% %% %% %% %% %% %% %% %% %% %% %% %% 

\section{Comparison of TOPSCAN with other rapid methods}
Two methods similar in principle to TOPSCAN are FAST-SSAP and the
constraint-programming methods based on TOPS
diagrams\cite{David_Gilbert_David_Westhead}.  FAST-SSAP uses
double dynamic programming to align vectors representing the secondary
structure elements to compare two structures. The constraint-based
method uses a 4-letter alphabet to represent the two types of
secondary structure going either up or down and adds information about
chirality and hydrogen bonds between secondary structure elements.
Comparison is performed by finding a maximum common template and
scoring the two structures against that template.  The approach taken
by TOPSCAN is unlikely to give results as good as either of these two
methods since the representation is so much simpler. However, this
does allow a very significant speed advantage. It is approximately
25$\times$ faster than the current implementation of FAST-SSAP and
10$\times$ faster than constraint-programming.  FAST-SSAP could be
speeded significantly by pre-calculating all the secondary structure
vectors and by calculating these vectors simply from the end-points of
the secondary structure elements rather than using an Eigenvector.

%% %% %% %% %% %% %% %% %% %% %% %% %% %% %% %% %% %% %% %% %% %% %% %% 
%% Check the timings                                                 %%
%% %% %% %% %% %% %% %% %% %% %% %% %% %% %% %% %% %% %% %% %% %% %% %% 

Compared with FAST-SSAP, the simple segregation of secondary structure
vectors into six quadrants throws away a lot of more detailed
information about the relative orientation of the elements, but as a
result, allows us to perform normal single dynamic programming
removing the need for double dynamic programming.

Compared with the constraint-programming method, our representation
actually has more information about directionality of the secondary
structure vectors, but loses the proximity information (encoded in the
constraint programming by hydrogen-bond information) and the chirality
arcs. 

\section{Conclusion}
TOPSCAN has proved a useful addition to the available methods for
comparing protein structures. In particular, it provides a useful
rapid method of ranking structures for further more detailed
comparison using SSAP. In common with the fast version of SSAP which
relies on secondary structure vectors, TOPSCAN is most likely to fail
when a structure is poorly defined and DSSP or STRIDE are unable to
make reliable secondary structure assignments.

\section{Acknowledgements}
I would like to thank Christine Orengo and Janet Thornton for making
the CATH data available and for their support during the first part of
this work which was funded by departmental funds at UCL as part of the
development of the CATHServer. 

\end{document}



