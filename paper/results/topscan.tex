\documentclass{article}
\usepackage{a4}
\usepackage{rotating}
\usepackage{epsfig}

\newcommand{\degrees}{\mbox{${}^{\circ}$}}

\title{Topscan --- A Tool for Rapid Comparison of Protein Topologies}
\author{Andrew C.R.\ Martin}

\begin{document}
\maketitle

\begin{abstract}
A novel and very simple algorithm is presented for comparing protein
topologies. In essence, the structure of the protein is reduced to a
string of letters encoding secondary structure and direction
information using a 12-letter alphabet. A simple dynamic programming
algorithm is then applied to compare two such strings and a score for
the similarity is calculated. A library of these strings may be built
from the Protein Databank (or a representative subset) and a new
structure may then be scanned against this library. The algorithm is
extremely fast, with a scan of a large TIM-barrel domain against a
library of more than 2000 secondary structure strings completing in
around 30 seconds on a 200MHz Pentium class processor.
\end{abstract}

\section{Introduction}
As new structure data becomes available, the classification of protein
folds is becoming more and more important. When a new structure is
solved, one wishes to ask the question of whether this fold has been
seen before. If the fold is one of the commonly occuring superfolds
(e.g.\ immunoglobulin-fold, TIM barrel, $\alpha\beta$-plait, Rossman fold),
then it can generally be recognised by eye. Automated servers to
answer this question have recently been developed and rely on
structure comparison programs. 

Unfortunately, detailed structure comparison is a time-consuming
process. The double-dynamic programming algorithm used in SSAP is
particularly computer intensive making it impractical to run a full
scan of a protein of any size. Our CATH-server has made use of
sequence screening and a hierachical expension scheme through the
representative levels in the CATH domain classification database to
reduce the search, but in the worst case scenario, it could still be
necessary to scan against all 2028 near-identical sequence
representatives. We estimate approximately 2 weeks of computer time
would be required to scan a large protein domain such as a TIM barrel
in this way and this is clearly impractical for use as a server.

This has led to the development of a novel simple algorithm for
reducing the search space.

\section{Methods}
\subsection{Creating topology strings}
The DSSP algorithm of Kabsch and Sander is used to assign secondary
structure information to a protein. From these assignments, regions of
$\beta$-sheet (Kabsch and Sander assignment, E) and of $\alpha$-helix
(Kabsch and Sander assignment, H) are extracted. Only continuous
regions of at least 3 or 4 residues with the same assignment are
selected (this parameter may be defined separately for helix and
strand). The topology of the protein is thus reduced to a string of E
and H characters.

To increase the information content of the topology string, the
end-points of each secondary structure element so-identified are found
and the vector between them is calculated. The direction of the vector
is grouped into one of 6 classes depending on the largest component of
the vector (i.e.\ positive or negative $x$, $y$, or $z$). This is
equivalent to saying the element points up, down, left, right, forward,
or back. The encoding is summarised in Table~\ref{tab:encoding}.

\begin{table}
\begin{center}
\begin{tabular}{llll} \hline
\multicolumn{2}{c}{Direction} & \multicolumn{2}{c}{Secondary Structure}\\ \cline{3-4}
          &         & strand & helix  \\ \hline
$+y$      & Up      & A      & G      \\
$+x$      & Right   & B      & H      \\
$-y$      & Down    & C      & I      \\
$-x$      & Left    & D      & J      \\
$+z$      & Back    & E      & K      \\
$-z$      & Forward & F      & L      \\ \hline
\end{tabular}
\end{center}
\caption{\label{tab:encoding}Encoding scheme used to represent
          secondary structure and direction information}
\end{table}

\subsection{Alignment}
Two topology strings are compared using a simple Needleman and Wunsch
dynamic programming algorithm. A scoring matrix is used in the
comparison based on the scores shown in Table~\ref{tab:matrix}. In
essence, the same secondary structure in the same orientation score
highest. Because of the boolean definition of whether a vector is in a
given quadrant, it is possible that vectors actually point in very
similar directions although they are in different quadrants (for
example one points at $+$89\degrees\ while another points at
$+$91\degrees). Vectors which are off by only one segment therefore
also score highly.

\begin{table}
\begin{center}
\begin{tabular}{lll}\hline
                        & \multicolumn{2}{c}{Secondary structure} \\ \cline{2-3}
Orientation             & Same  & Different     \\ \hline
Same                    & 10    & 3             \\
Off by 1 quadrant       & 8     & 1             \\
Off by 2 quadrants      & 2     & 0             \\ \hline
\end{tabular}
\end{center}
\caption{\label{tab:matrix} Scoring scheme employed in the matrix for
the dynamic programming comparison of two topology strings.}
\end{table}

Because any pair of proteins is in an arbitrary relative orientation,
the definition of ``up'' in one protein may not correspond to ``up''
in the second. Therefore, one of the strings is permuted 23 times,
such that the dynamic programming algorithm comparison is performed a
total of 24 times (the 6 sides of a cube, each of which may be 4 ways
up). Table~\ref{tab:permute} shows the modifications made to the
encoding to achieve rotations about $x$, $y$ and $z$ axes.

\begin{table}
\begin{center}
\begin{tabular}{llll}\hline
        & \multicolumn{3}{c} {Rotation axis} \\ \cline{2-4}
        & $x$           & $y$           & $z$           \\ \hline
Old     & ABCDEFGHIJKL  & ABCDEFGHIJKL  & ABCDEFGHIJKL  \\
New     & EBFDCAKHLJIG  & AECFDBGKILJH  & BCDAEFHIJGKL  \\ \hline
\end{tabular}
\end{center}
\caption{\label{tab:permute} Modifications made to topology strings to
        achieve rotations about the $x$, $y$ and $z$ axes.}
\end{table}




\section{Results and Discussion}
\input results.tex
\input best.tex
\input worst.tex
\input bars.tex
\subsection{General Results}


\subsection{Effect of Secondary Structure Length}


\subsection{Problems}
The algorithm has problems in discriminating, for example, a TIM
barrel from a Rossmann fold. These are topologically somewhat similar
--- in both cases, they consist of a core of $\beta$-sheet with
helices on the outside. In the case of the TIM barrel the
$\beta$-sheet is curved into a barrel where as the Rossmann fold has a
relatively flat sheet. Because the differences between the two occur
in a plane perpendicular to the direction of the secondary structure
elements, the current method does not distinguish reliably between these
architectures. 

The assignment of secondary structure is the critical first stage. We
have found that the DSSP algorithm can be over-sensitive to errors in
the structure. For example, NMR structures often show little secondary
structure in DSSP assignments whereas an intuitive visual inspection
shows substantial secondary structure. The simplified secondary
structure assignment scheme used within Rasmol actually appears to
perform better in assigning secondary structures in an intuitive
manner. 


\end{document}

